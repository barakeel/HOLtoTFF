\documentclass{beamer}
\usepackage{etex}
\usepackage{booktabs}

\usepackage{multicol}
\usepackage{calc}
\usepackage{ifthen}
\usepackage{beamerthemeshadow}
\setbeamertemplate{navigation symbols}{}
%packages indispensables 
\usepackage[utf8]{inputenc}
\usepackage{lmodern}
%packages utiles
\usepackage{alltt} %program code
\usepackage{enumerate}
\usepackage{amssymb} %lettres mathématiques
\usepackage{amsmath}
\usepackage{amsthm}
\usepackage{bussproofs} %derivation
\usepackage{hyperref} %to write path.
\usepackage{color} % colouring text
\usepackage{tabularx} % table
\usepackage{graphicx}
\usepackage{caption}
\usepackage{subcaption}
\setbeamertemplate{headline}{}
{\renewcommand{\arraystretch}{1.5}

%%%%%%%%%%%%%%%%% graphics %%%%%%%%%%%%%%%%%%%%%%
\usepackage{pgfplots}
\usepackage{pgf-pie}
\usepackage{tikz} % to draw

\usetikzlibrary{shapes,arrows}
\usetikzlibrary{trees,positioning,fit}
\tikzstyle{decision} = [diamond, draw, fill=blue!20, 
    text width=4.5em, text badly centered, node distance=3cm, inner sep=0pt]
\tikzstyle{block} = [rectangle, draw, fill=blue!20, 
    text width=5em, text centered, rounded corners, minimum height=4em]
\tikzstyle{line} = [draw, -latex']
\tikzstyle{cloud} = [draw, ellipse,fill=red!20, node distance=3cm,
    minimum height=2em]
%%%%%%%%%%%%%%%%%%%%%%%%%%%%%%%%%%%%%%%%%%%%%%%%%%
\begin{document}

\title{Beagle as a HOL4 external ATP method}  
\author{Thibault Gauthier}
\date{\today} 

\frame{\titlepage} 

\begin{frame}
  \tableofcontents
\end{frame}

\section{Introduction}
\subsection{Two types of provers}
\frame{\frametitle{Two types of provers} 

\noindent \begin{tabularx}{\textwidth}{ |c|X|X| }
  \hline & HOL4 & Beagle\\
  \hline  Type & Interactive & Automated \\
  \hline Expressivity & Higher-order & First-order\\  
  \hline Soundness & Small kernel (LCF) & Long optimized code \\
  \hline Family & HOL Light, ProofPower, Isabelle/HOL &  Spass + T \\
  \hline
\end{tabularx}
}

\subsection{Problem statement} 
\frame{\frametitle{Problem statement} 

\textit{Problem} Here are two HOL4 internal provers.
\begin{enumerate}
\item[-] Metis: first-order 
\item[-] Cooper: arithmetic 
\end{enumerate}
\pause
\textit{Solution} An external prover.
 \begin{enumerate}
  \item[-] Beagle: first-order and arithmetic
\end{enumerate}
}

\subsection{Interaction}
\frame{\frametitle{Interaction} 
\begin{tikzpicture}[node distance = 2cm, auto]
  % Place nodes
  \node [cloud] (conjecture) {conjecture};
  \node [block, right of=conjecture,  node distance=3cm] (HOL1) 
  {translation};
  \node [block, right of=HOL1, node distance=3cm] (TFF1) 
  {problem's file};
  \node [block, right of=TFF1, node distance=3cm, yshift=-1cm] (Beagle) 
  {automatic proof};
  \node [block, below of=TFF1, node distance=2cm] (TFF2) 
  {proof's file};  
  \node [block, dashed, below of=HOL1, node distance=2cm] (HOL2) 
  {replaying the proof};
  \node [cloud, below of=conjecture, node distance=2cm] (theorem) 
  {theorem};
  \node [label=HOL4, draw=black, ultra thick, 
  fit=(conjecture) (theorem) (HOL1) (HOL2)] {}; 
  \node [label=format TFA, draw=black, ultra thick, fit=(TFF1) (TFF2)] 
  {}; 
  \node [label=Beagle, draw=black, ultra thick, fit=(Beagle)] {}; 
  % Draw edges
  \draw [-to,black,ultra thick] (conjecture) -- (HOL1);
  \draw [-to,black,ultra thick] (HOL1) -- (TFF1);
  \draw [-to,,black,ultra thick] (TFF1) -- (Beagle);
  \draw [-to,black,ultra thick] (Beagle) -- (TFF2);
  \draw [-to,black,ultra thick] (TFF2) -- (HOL2);
  \draw [-to,black,ultra thick,dashed] (HOL2) -- (theorem);
  \draw [-to,black,ultra thick] (Beagle.south) to [out=225,in=330](theorem.south);
\end{tikzpicture}
}



\section{Translation to first-order}


\frame{\frametitle{Translation's order}
\begin{enumerate}
  \item {\color{blue} Monomorphization}
  \item Negation of the conclusion
  \item Rewriting to conjunctive normal form
  \item {\color{blue} $\lambda$-lifting}
  \item Boolean argument elimination: $P(f) \ \rightarrowtail \ f \Rightarrow P(T) \wedge \neg f \Rightarrow P(F)$
  \item Rewriting to a clause set
  \item {\color{blue} Defunctionalization}
  \item Mapping numeral to integers
\end{enumerate}
}

\subsection{Monomorphization}
\frame{\frametitle{Monomorphization}
Instantiation of polymorphic types ($a$,$\ldots$).
 \begin{exampleblock}{Problem}
Thm 1: $\forall x:a.\ {\color{violet!60}D} \ x\ 0$ \ \ 
Thm 2: ${\color{violet!60}C} = \lambda x:a.\ {\color{violet!60}D}\ x\ 0$
\\Conjecture : ${\color{orange}C} \ 2$
 \end{exampleblock}
\pause
 \begin{exampleblock}{Matching type of ${\color{violet!60}C} : a \rightarrow bool$  and ${\color{orange}C}: int \rightarrow bool$ }
Thm 1: $\forall x:a.\ {\color{violet!60}D} \ x\ 0$ \ \ 
Thm 2: ${\color{orange}C} = \lambda x:int.\ {\color{orange}D}\ x\ 0$
 \\Conjecture: ${\color{orange}C}\ 2$
\end{exampleblock}
\pause
 \begin{exampleblock}{Matching type of ${\color{violet!60}D} : a \rightarrow  int \rightarrow bool$  and ${\color{orange}D}: int \rightarrow int \rightarrow bool$ }
Thm 1: $\forall x:int.\ {\color{orange}D} \ x\ 0$ \ \ 
Thm 2: ${\color{orange}C} = \lambda x:int.\ {\color{orange}D}\ x\ 0$
 \\Conjecture: ${\color{orange}C}\ 2$
\end{exampleblock}
}


\subsection{$\lambda$-lifting}
\frame{\frametitle{$\lambda$-lifting}
 \begin{exampleblock}{Problem}
Thm 1: $\forall x.\ D\ x\ 0$ \ \ 
Thm 2: $C = \lambda x.\ D\ x\ 0$
\\Conjecture: $C\ 2$
\end{exampleblock}
\pause
 \begin{exampleblock}{Negation of the conclusion}
\[\lbrace \forall x.\ D\ x\ 0 \ , \  
C = \lambda x.\ D\ x\ 0 \ , \  
\neg (C\ 2) \rbrace\]
\end{exampleblock}

\pause
$\lambda$-lifting: $\exists {\color{blue}Gen}.\ (\forall x.\ {\color{blue}Gen}\ x = D\ x\ 0) \wedge C = {\color{blue}Gen} $ \\
\pause
Combinators: $C = S\ (S\ (K\ D)\ I)\ (K\ 0) $
\pause

\begin{exampleblock}{Clause set}
\[\lbrace 
\forall x.\ D\ x\ 0 \ , \ 
\forall x.\ Gen\ x = D\ x\ 0 \ , \ 
C = Gen \ , \ 
\neg (C\ 2) \rbrace \]
\end{exampleblock}
}


\subsection{Defunctionalization}
\frame{\frametitle{Defunctionalization}
Let ${\color{blue}App}$ be the apply functor verifying ${\color{blue}App}\ f\ x$ = $f\ x$ ($ f\ x \rightarrowtail {\color{blue}App}\ f\ x$).
 We defunctionalize a function only when: 

 \begin{enumerate}
 \item [-] it is not a mapped function
 \pause
 \item [-] it is quantified universally
   $!h.\ h\ x\ y \ \rightarrowtail \ !h.\ {\color{blue}App}\ ({\color{blue}App}\ h\ x)\ y$
 \pause
 \item [-] it is used with different number of arguments
   $\lbrace 
   h\ x\ y\ z \ , \ h\ x = j 
   \rbrace  \ \rightarrowtail \ \lbrace 
   {\color{blue}App}\ ({\color{blue}App}\ (h\ x)\ y)\ z \ , 
   \ h\ x = j
   \rbrace$
  \item [-] it has the same type as an universally quantified function
  \pause
\end{enumerate}
\pause

\begin{exampleblock}{Defunctionalization}
\[\lbrace 
\forall x.\ D\ x\ 0, \ 
\forall x.\ Gen\ x = D\ x\ 0, \ 
C = Gen, \ 
\neg (C\ 2) \rbrace \]
\[\lbrace 
\forall x.\ D\ x\ 0, \ 
\forall x.\ {\color{blue}App}\ Gen\ x = D\ x\ 0, \ 
C = Gen, \ 
\neg (C\ 2) \rbrace \]
\end{exampleblock}
}

\frame{\frametitle{Translation's order}
\begin{enumerate}
  \item {\color{blue} Monomorphization}
  \item Negation of the conclusion
  \item Rewriting to conjunctive normal form
  \item {\color{blue} $\lambda$-lifting}
  \item {Boolean argument elimination}
  \item Rewriting to a clause set
  \item {\color{blue} Defunctionalization}
  \item Mapping numeral to integers
\end{enumerate}
}

\section{Demonstration}
\frame{\frametitle{Demonstration}
 \begin{exampleblock}{Problem}
Thm 1: $\forall x:a.\ {\color{violet!60}D} \ x\ 0$ \ \ 
Thm 2: ${\color{violet!60}C} = \lambda x:a.\ {\color{violet!60}D}\ x\ 0$
\\Conjecture : ${\color{orange}C} \ 2$
\end{exampleblock}
\begin{exampleblock}{Translated problem}
\[
\lbrace 
\forall x:a.\ {\color{violet!60}D}\ x\ 0, \ 
\forall x:a.\ App\ Gen_0\ x = {\color{violet!60}D}\ x\ 0, \ 
{\color{violet!60}C} = Gen_0, \  
\] \[
\forall x:int.\ {\color{orange}D}\ x\ 0, \ 
\forall x:int.\ App\ Gen_1\ x = {\color{orange}D}\ x\ 0, \ 
{\color{orange}C} = Gen_1, \ 
\] \[ \neg ({\color{orange}C}\ 2) 
\rbrace 
\]
\end{exampleblock} 
}



\section{Results}
\frame{\frametitle{Results}
\begin{table}[]
\begin{tabular}{clc}
 \begin{tikzpicture}[scale = 0.5]
 \pie[text=inside, color ={green!90, black!40, black!30, black!20}]
 {70/,15/,7/,8/}   
 \end{tikzpicture}
 &  &
 \begin{tikzpicture}[scale = 0.5, yshift = 3]
 \pie[text=legend,color ={green!90, black!40, black!30, black!20}]
 {81/unsat,2/sat,8/unknown,9/time out}   
 \end{tikzpicture}
\end{tabular}
\caption {With and without monomorphization}
\end{table}
Only $56 \%$ of higher-order problems are solved. \\
And $79 \%$ of arithmetic problems are solved. (containing at least one mapped arithmetic constant)
}



\section{Conclusion}
\subsection{}
\frame{\frametitle{Summary on the current HOL4-Beagle interaction}
Qualities: 
\begin{enumerate}
\item [-] Its translation is correct (preserve satisfiability).
\item [-] It proves 81\% of conjectures solved by Metis without arithmetic lemmas.
\item [-] It uses a well-known format, TFA (TPTP).

\end{enumerate} 
Limitations: 
\begin{enumerate}
\item [-] it is incomplete (doesn't preserve unsatisfiability).
\item [-] it doesn't reconstruct the proof.
\item [-] it doesn't support real and rational arithmetic.
\item [-] it was not tested extensively (for example, as part of a lemma mining method).
\end{enumerate}
}
\end{document}

