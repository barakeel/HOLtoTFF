\documentclass{beamer}
\usepackage{etex}


\usepackage{multicol}
\usepackage{calc}
\usepackage{ifthen}
\usepackage{beamerthemeshadow}
\setbeamertemplate{navigation symbols}{}
%packages indispensables 
\usepackage[utf8]{inputenc}
\usepackage{lmodern}
\usepackage{graphicx}
%packages utiles
\usepackage{alltt} %program code
\usepackage{enumerate}
\usepackage{amssymb} %lettres mathématiques
\usepackage{amsmath}
\usepackage{amsthm}
\usepackage{bussproofs} %derivation
\usepackage{hyperref} %to write path.
\usepackage{color} % colouring text
\usepackage{tabularx} % table
{\renewcommand{\arraystretch}{1.5}

\usepackage{fancyvrb}
\usepackage{pgfplots}
%%%%%%%%%%%%%%%%% graphics %%%%%%%%%%%%%%%%%%%%%%
\usepackage{tikz} % to draw
\usetikzlibrary{shapes,arrows}
\usetikzlibrary{trees,positioning,fit}
\tikzstyle{decision} = [diamond, draw, fill=blue!20, 
    text width=4.5em, text badly centered, node distance=3cm, inner sep=0pt]
\tikzstyle{block} = [rectangle, draw, fill=blue!20, 
    text width=5em, text centered, rounded corners, minimum height=4em]
\tikzstyle{line} = [draw, -latex']
\tikzstyle{cloud} = [draw, ellipse,fill=red!20, node distance=3cm,
    minimum height=2em]

\newcommand{\slice}[5]{
  \pgfmathparse{0.5*#1+0.5*#2}
  \let\midangle\pgfmathresult
  % slice
  \draw[thick,fill=#5] (0,0) -- (#1:1) arc (#1:#2:1) -- cycle;

  % outer label
  \node[label=\midangle:#4] at (\midangle:1) {};

  % inner label
  \pgfmathparse{min((#2-#1-10)/110*(-0.3),0)}
  \let\temp\pgfmathresult
  \pgfmathparse{max(\temp,-0.5) + 0.8}
  \let\innerpos\pgfmathresult
  \node at (\midangle:\innerpos) {#3};
}
%%%%%%%%%%%%%%%%%%%%%%%%%%%%%%%%%%%%%%%%%%%%%%%%%%
\begin{document}

\title{HOL4-Beagle, de l'ordre supérieur vers le premier ordre}  
\author{Thibault Gauthier}
\date{\today} 

\frame{\titlepage} 

\section{Introduction}
\subsection{Deux types de prouveurs}
\frame{\frametitle{Deux façon différentes de démontrer des théorèmes} 

\noindent \begin{tabularx}{\textwidth}{ |c|X|X| }
  \hline
  & Prouveur interactif & Prouveur automatique\\
  \hline  
  Prouveurs & HOL4, Coq, \ldots 
  & Beagle, SPASS, \ldots \\
  \hline  Expressivité & Ordre supérieur. & Premier ordre.
  \\  
  \hline Efficacité & Guidé. & Automatique. \\
  \hline Sûreté & Petit noyau.
 & Code assez long. \\
  \hline
\end{tabularx}
}

\begin{frame}
  \tableofcontents
\end{frame}

\subsection{\'Enoncé du problème} 
\frame{\frametitle{\'Enoncé du problème} 

\textit{Problème} 
Voilà deux prouveurs internes à HOL4.
\begin{enumerate}
\item[-] Metis: ordre supérieur
\item[-] Cooper: arithmétique
\end{enumerate}
\pause
\textit{Solution} Un prouveur externe.
 \begin{enumerate}
  \item[-] Beagle: premier ordre et arithmétique 
\end{enumerate}
}

\subsection{Schéma d'interaction}
\frame{\frametitle{Schéma d'interaction} 
\begin{tikzpicture}[node distance = 2cm, auto]
  % Place nodes
  \node [cloud] (conjecture) {conjecture};
  \node [block, right of=conjecture,  node distance=3cm] (HOL1) 
  {traduction};
  \node [block, right of=HOL1, node distance=3cm] (TFF1) 
  {fichier du problème};
  \node [block, right of=TFF1, node distance=3cm, yshift=-1cm] (Beagle) 
  {Preuve automatique};
  \node [block, below of=TFF1, node distance=2cm] (TFF2) 
  {fichier de la preuve};  
  \node [block, dashed, below of=HOL1, node distance=2cm] (HOL2) 
  {construction de la preuve};
  \node [cloud, below of=conjecture, node distance=2cm] (theorem) 
  {théorème};
  \node [label=HOL4, draw=black, ultra thick, 
  fit=(conjecture) (theorem) (HOL1) (HOL2)] {}; 
  \node [label=format TFF, draw=black, ultra thick, fit=(TFF1) (TFF2)] 
  {}; 
  \node [label=Beagle, draw=black, ultra thick, fit=(Beagle)] {}; 
  % Draw edges
  \draw [-to,black,ultra thick] (conjecture) -- (HOL1);
  \draw [-to,black,ultra thick] (HOL1) -- (TFF1);
  \draw [-to,,black,ultra thick] (TFF1) -- (Beagle);
  \draw [-to,black,ultra thick] (Beagle) -- (TFF2);
  \draw [-to,black,ultra thick] (TFF2) -- (HOL2);
  \draw [-to,black,ultra thick,dashed] (HOL2) -- (theorem);
  \draw [-to,black,ultra thick] (Beagle.south) to [out=225,in=330](theorem.south);
\end{tikzpicture}
}



\section{Traduction vers le premier ordre}


\frame{\frametitle{Ordre de la traduction vers le premier ordre}
\begin{enumerate}
  \item {\color{blue} Monomorphisation}
  \item Négation de la conclusion
  \item Mise en forme normale conjonctive
  \item {\color{blue} $\lambda$-lifting}
  \item {\color{blue} Elimination des booléens}
  \item Mise sous forme d'un ensemble de clauses
  \item {\color{blue} Défonctionnalisation}
  \item Injection des numéraux dans les entiers
  \item Instantiation des variables booléennes quantifiées
\end{enumerate}
}

\subsection{Monomorphisation}
\frame{\frametitle{Monomorphisation}
Instanciation des types polymorphes ($a$,$b$,$\ldots$) de HOL4 par des types monomorphes ($int$,$bool$,$\ldots$). 
 \begin{exampleblock}{Problème}
Théorème 1: $\forall y:b.\ {\color{violet!60}D} \ y$ \ \ 
Théorème 2: $\forall x:a .\ {\color{blue!60}D}\ x \Rightarrow {\color{blue!60}C}\ x$
\\Conjecture : ${\color{orange}C} \ 2$
 \end{exampleblock}

 \begin{exampleblock}{Unification du type de ${\color{blue!60}C} : a  \rightarrow bool$  et de ${\color{orange}C}: int \rightarrow bool$ }
Théorème 1: $\forall y:b.\ {\color{violet!60}D} \ y$ \ \ 
Théorème 2: $\forall x:num.\ {\color{orange}D}\ x \Rightarrow {\color{orange}C}\ x$
 \\Conjecture: ${\color{orange}C}\ 2$
\end{exampleblock}

 \begin{exampleblock}{Unification du type de ${\color{violet!60}D} : b  \rightarrow bool$  et de ${\color{orange}D}: num \rightarrow bool$ }
Théorème 1: $\forall y:num.\ {\color{orange}D} \ y$ \ \ 
\\Théorème 2: $\forall x:num.\ {\color{orange}D}\ x \Rightarrow {\color{orange}C}\ x$
\\Conjecture: ${\color{orange}C}\ 2$
\end{exampleblock}
}

\subsection{$\lambda$-lifting}
\frame{\frametitle{$\lambda$-lifting}                    
\[ P\ (\lambda x.x+1) \]
\[ \exists f.\ (\forall x.\ f\ x = x + 1) \wedge P\ f \]
}

\subsection{Elimination des booléens}
\frame{\frametitle{Elimination des booléens}

\[P (\forall x.\ x=0)\]
\[((\forall x.\ x=0) \Rightarrow P\ true) \wedge 
   (\neg (\forall x.\ x=0)  \Rightarrow P\ false)\]
}

\subsection{Défonctionnalisation}
\frame{\frametitle{Défonctionnalisation}
Soit $App$ vérifiant $App\ f\ x$ = $f\ x$. On effectue une défonctionnalisation lorsqu'une fonction non-arithmétique: 

 \begin{enumerate}
 \item [-] est quantifiée universellement
   \[!h.\ h\ x\ y = 0\]
   \[!h.\ App\ (App\ h\ x)\ y = 0\]
 \item [-] a le même type qu'une fonction quantifiée universellement
 \item [-] a un nombre d'arguments qui varie
   \[h\ x\ y\ z\wedge h\ x = g\]
   \[App\ (App\ (h\ x)\ y)\ z \wedge h\ x = g\]
 \end{enumerate}.
}


\section{Conclusion}
\subsection{Qualités et limites}
\frame{\frametitle{Qualités et limites de l'interaction HOL4-Beagle}
Qualités: 
\begin{enumerate}
\item [-] Résout des problèmes arithmétiques sans guidage
\item [-] Utilise un format de communication répandu
\item [-] Est correcte et préserve l'insatisfaisabilité
\end{enumerate} 
Limites: 
\begin{enumerate}
\item [-] 20\% des conjectures prouvées par Metis ne sont pas prouvées par Beagle.
\item [-] Est incomplète et ne préserve pas la satisfaisabilité
\item [-] Ne génére pas automatiquement des théorèmes aidant à prouver la conjecture
\item [-] Ne rejoue pas la preuve
\end{enumerate} 
}

\end{document}

