\documentclass{beamer}
\usepackage{etex}


\usepackage{multicol}
\usepackage{calc}
\usepackage{ifthen}
\usepackage{beamerthemeshadow}
\setbeamertemplate{navigation symbols}{}
%packages indispensables 
\usepackage[frenchb]{babel}
\usepackage[utf8]{inputenc}
\usepackage{lmodern}
\usepackage{graphicx}
%packages utiles
\usepackage{alltt} %program code
\usepackage{enumerate}
\usepackage{amssymb} %lettres mathématiques
\usepackage{amsmath}
\usepackage{amsthm}
\usepackage{bussproofs} %derivation
\usepackage{hyperref} %to write path.
\usepackage{color} % colouring text
\usepackage{tabularx} % table
{\renewcommand{\arraystretch}{1.5}

\usepackage{fancyvrb}
\usepackage{pgfplots}
%%%%%%%%%%%%%%%%% graphics %%%%%%%%%%%%%%%%%%%%%%
\usepackage{tikz} % to draw
\usetikzlibrary{shapes,arrows}
\usetikzlibrary{trees,positioning,fit}
\tikzstyle{decision} = [diamond, draw, fill=blue!20, 
    text width=4.5em, text badly centered, node distance=3cm, inner sep=0pt]
\tikzstyle{block} = [rectangle, draw, fill=blue!20, 
    text width=5em, text centered, rounded corners, minimum height=4em]
\tikzstyle{line} = [draw, -latex']
\tikzstyle{cloud} = [draw, ellipse,fill=red!20, node distance=3cm,
    minimum height=2em]


%%%%%%%%%%%%%%%%%%%%%%%%%%%%%%%%%%%%%%%%%%%%%%%%%%
\begin{document}

\title{HOL4-Beagle, de l'ordre supérieur vers le premier ordre}  
\author{Thibault Gauthier}
\date{\today} 

\frame{\titlepage} 

\section{Introduction}
\subsection{Deux types de prouveurs}
\frame{\frametitle{Deux types de prouveurs} 

\noindent \begin{tabularx}{\textwidth}{ |c|X|X| }
  \hline
  & HOL4 & Beagle\\
  \hline  
  Type & Interactif & Automatique \\
  \hline Expressivité & Ordre supérieur & Premier ordre\\  
  \hline Sûreté & Petit noyau
 & Code assez long \\
  \hline
\end{tabularx}
}

\subsection{\'Enoncé du problème} 
\frame{\frametitle{\'Enoncé du problème} 

\textit{Problème} 
Voilà deux prouveurs internes à HOL4.
\begin{enumerate}
\item[-] Metis: premier ordre + traduction de l'ordre supérieur vers le premier ordre
\item[-] Cooper: arithmétique 
\end{enumerate}
\pause
\textit{Solution} Un prouveur externe.
 \begin{enumerate}
  \item[-] Beagle: premier ordre et arithmétique 
\end{enumerate}
}

\subsection{Schéma d'interaction}
\frame{\frametitle{Schéma d'interaction} 
\begin{tikzpicture}[node distance = 2cm, auto]
  % Place nodes
  \node [cloud] (conjecture) {conjecture};
  \node [block, right of=conjecture,  node distance=3cm] (HOL1) 
  {traduction};
  \node [block, right of=HOL1, node distance=3cm] (TFF1) 
  {fichier du problème};
  \node [block, right of=TFF1, node distance=3cm, yshift=-1cm] (Beagle) 
  {Preuve automatique};
  \node [block, below of=TFF1, node distance=2cm] (TFF2) 
  {fichier de la preuve};  
  \node [block, dashed, below of=HOL1, node distance=2cm] (HOL2) 
  {construction de la preuve};
  \node [cloud, below of=conjecture, node distance=2cm] (theorem) 
  {théorème};
  \node [label=HOL4, draw=black, ultra thick, 
  fit=(conjecture) (theorem) (HOL1) (HOL2)] {}; 
  \node [label=format TFF, draw=black, ultra thick, fit=(TFF1) (TFF2)] 
  {}; 
  \node [label=Beagle, draw=black, ultra thick, fit=(Beagle)] {}; 
  % Draw edges
  \draw [-to,black,ultra thick] (conjecture) -- (HOL1);
  \draw [-to,black,ultra thick] (HOL1) -- (TFF1);
  \draw [-to,,black,ultra thick] (TFF1) -- (Beagle);
  \draw [-to,black,ultra thick] (Beagle) -- (TFF2);
  \draw [-to,black,ultra thick] (TFF2) -- (HOL2);
  \draw [-to,black,ultra thick,dashed] (HOL2) -- (theorem);
  \draw [-to,black,ultra thick] (Beagle.south) to [out=225,in=330](theorem.south);
\end{tikzpicture}
}

\begin{frame}
  \tableofcontents
\end{frame}

\section{Traduction vers le premier ordre}


\frame{\frametitle{Ordre de la traduction vers le premier ordre}
\begin{enumerate}
  \item {\color{blue} Monomorphisation}
  \item Négation de la conclusion
  \item Mise en forme normale conjonctive
  \item {\color{blue} $\lambda$-lifting}
  \item {\'Elimination des booléens}
  \item Mise sous forme d'un ensemble de clauses
  \item {\color{blue} Défonctionnalisation}
\end{enumerate}
}

\subsection{Monomorphisation}
\frame{\frametitle{Monomorphisation}
Instanciation des types polymorphes ($a$,$\ldots$) par des types monomorphes ($int$,$bool$,$\ldots$). 
 \begin{exampleblock}{Problème}
Thm 1: $\forall x:a.\ {\color{violet!60}D} \ x\ 0$ \ \ 
Thm 2: ${\color{violet!60}C} = \lambda x:a.\ {\color{violet!60}D}\ x\ 0$
\\Conjecture : ${\color{orange}C} \ 2$
 \end{exampleblock}
\pause
 \begin{exampleblock}{Unification de ${\color{violet!60}C} : a \rightarrow int \rightarrow bool$  et de ${\color{orange}C}: int \rightarrow int \rightarrow bool$ }
Thm 1: $\forall x:a.\ {\color{violet!60}D} \ x\ 0$ \ \ 
Thm 2: ${\color{orange}C} = \lambda x:int.\ {\color{orange}D}\ x\ 0$
 \\Conjecture: ${\color{orange}C}\ 2$
\end{exampleblock}
\pause
 \begin{exampleblock}{Unification de ${\color{violet!60}D} : a \rightarrow  int \rightarrow bool$  et de ${\color{orange}D}: int \rightarrow int \rightarrow bool$ }
Thm 1: $\forall x:int.\ {\color{orange}D} \ x\ 0$ \ \ 
Thm 2: ${\color{orange}C} = \lambda x:int.\ {\color{orange}D}\ x\ 0$
 \\Conjecture: ${\color{orange}C}\ 2$
\end{exampleblock}
}


\subsection{$\lambda$-lifting}
\frame{\frametitle{$\lambda$-lifting}
 \begin{exampleblock}{Problème}
Thm 1: $\forall x.\ D\ x\ 0$ \ \ 
Thm 2: $C = \lambda x.\ D\ x\ 0$
\\Conjecture: $C\ 2$
\end{exampleblock}
 \begin{exampleblock}{Négation de la conclusion}
\[\lbrace \forall x.\ D\ x\ 0 \ , \  
C = \lambda x.\ D\ x\ 0 \ , \  
\neg (C\ 2) \rbrace\]
\end{exampleblock}

\pause
$\lambda$-lifting :
\[C = \lambda x.\ D\ x\ 0 \ \rightarrowtail \
 \exists {\color{blue}f}.\ (\forall x.\ {\color{blue}f}\ x = D\ x\ 0) \wedge C = {\color{blue}f} \]


\begin{exampleblock}{Mise sous forme d'un ensemble de clauses}
\[\lbrace 
\forall x.\ D\ x\ 0 \ , \ 
\forall x.\ f\ x = D\ x\ 0 \ , \ 
C = f \ , \ 
\neg (C\ 2) \rbrace \]
\end{exampleblock}

}


\subsection{Défonctionnalisation}
\frame{\frametitle{Défonctionnalisation}
Soit ${\color{blue}App}$ vérifiant ${\color{blue}App}\ f\ x$ = $f\ x$. On effectue une défonctionnalisation lorsqu'une fonction non-arithmétique : 

 \begin{enumerate}
 \item [-] est quantifiée universellement
   $!h.\ h\ x\ y \ \rightarrowtail \ !h.\ {\color{blue}App}\ ({\color{blue}App}\ h\ x)\ y$
 \item [-] a le même type qu'une fonction quantifiée universellement
 \item [-] a un nombre d'arguments auxquelles la fonction est appliquée qui varie
   $\lbrace 
   h\ x\ y\ z \ , \ h\ x = j 
   \rbrace  \ \rightarrowtail \ \lbrace 
   {\color{blue}App}\ ({\color{blue}App}\ (h\ x)\ y)\ z \ , 
   \ h\ x = j
   \rbrace$
\end{enumerate}
\pause

\begin{exampleblock}{Défonctionnalisation}
\[\lbrace 
\forall x.\ D\ x\ 0 \ , \ 
\forall x.\ f\ x = D\ x\ 0 \ , \ 
C = f \ , \ 
\neg (C\ 2) \rbrace \]
\[\lbrace 
\forall x.\ D\ x\ 0 \ , \ 
\forall x.\ {\color{blue}App}\ f\ x = D\ x\ 0 \ , \ 
C = f \ , \ 
\neg (C\ 2) \rbrace \]
\end{exampleblock}

}



\section{Conclusion}
\subsection{Qualités et limites}
\frame{\frametitle{Qualités et limites de l'interaction HOL4-Beagle}
Qualités: 
\begin{enumerate}
\item [-] est correcte (préserve la satisfaisabilité)
\item [-] prouve 80\% des conjectures prouvées par Metis auxquelles on a enlevé les lemmes arithmétiques
\item [-] utilise un format de communication répandu

\end{enumerate} 
Limites: 
\begin{enumerate}
\item [-] est incomplète et (ne préserve pas l'insatisfaisabilité)
\item [-] ne cherche pas automatiquement des théorèmes aidant à prouver la conjecture
\item [-] ne rejoue pas (encore) la preuve
\end{enumerate} 
}

\end{document}

