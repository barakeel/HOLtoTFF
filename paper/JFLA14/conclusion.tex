\section{Conclusion}

Cet article présente une traduction de la logique d'ordre supérieure
polymorphe vers la logique du premier ordre monomorphe avec
arithmétique, faisant notamment intervenir un algorithme de
monomorphisation simple mais néanmoins efficace. Cette traduction est
implantée et prouvée correcte en \holfour, permettant à cet assistant de
preuve de bénéficier de prouveurs du premier ordre avec (ou sans)
arithmétique tout en ayant confiance dans la traduction. L'application à
\beagle permet de fournir une tactique utilisable et efficace à
\holfour, combinant raisonnement propositionnel et arithmétique
linéaire. Un premier prototype permet de vérifier les traces au format
\tff afin de garantir la correction, prototype également appliqué à
\beagle.

Le travail présenté ici constitue donc une base pour l'interaction entre
\holfour et des prouveurs externes, qui peut être encore largement
développée. Les résultats expérimentaux sont encourageants quant aux
perspectives offertes par une telle intégration. Ce travail constitue
également une première utilisation réelle de \beagle, montrant son
efficacité et son utilité.


\paragraph{Remerciements}

Les auteurs sont particulièrement reconnaissants envers Peter
Baumgartner et Josh Bax pour leurs explications sur le fonctionnement de
\beagle, ainsi que leur aide pour la résolution de certaines des erreurs
produites par des versions préliminaires de la traduction. Les auteurs
remercient également Alexis Saurin pour ses suggestions à propos de
cet article.
