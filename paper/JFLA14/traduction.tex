\section{Une traduction prouvée de l'un vers l'autre}
 
A mettre quelque part (Où trouver le code source?)
 \path{https://github.com/barakeel/HOLtoTFF}. Some examples can be found in paperexample.sml.

\subsection{Présentation}

Voici comment \beagletac fonctionne: (à déplacer à un autre endroit) 

\begin{tikzpicture}[node distance = 2cm, auto]
  % Place nodes
  \node [cloud] (problem) {problème};
  \node [block, right of=problem,  node distance=3cm] (HOL1) 
  {traduction};
  \node [block, right of=HOL1,  node distance=3cm] (HOLTFF1) 
  {impression};
  \node [block, right of=HOLTFF1, node distance=3cm] (TFF1) 
  {fichier problème};
  \node [block, right of=TFF1, node distance=3cm, yshift = -1cm] (Beagle) 
  {recherche de preuve};
  \node [block, below of=TFF1, node distance=2cm] (TFF2) 
  {fichier preuve};  
  \node [block, below of=HOLTFF1, node distance=2cm] (HOLTFF2) 
  {lecture};
  \node [block, below of=HOL1, node distance=2cm] (HOL2) 
  {rejouage};
  \node [cloud, below of=problem, node distance=2cm] (theorem) 
  {théorème};
  \node [label=HOL4, draw=black, ultra thick, 
  fit=(problem) (theorem) (HOLTFF1) (HOL1) (HOL2)] {}; 
  \node [label=TFF format interface, draw=black, ultra thick, fit=(TFF1) (TFF2)] 
  {}; 
  \node [label=Beagle, draw=black, ultra thick, fit=(Beagle)] {}; 
  % Draw edges
  \draw [-to,black,ultra thick] (problem) -- (HOL1);
  \draw [-to,black,ultra thick] (HOL1) -- (HOLTFF1);
  \draw [-to,black,ultra thick] (HOLTFF1) -- (TFF1);
  \draw [-to,,black,ultra thick] (TFF1) -- (Beagle);
  \draw [-to,black,ultra thick,dashed] (Beagle) -- (TFF2);
  \draw [-to,black,ultra thick,dashed] (TFF2) -- (HOLTFF2);
  \draw [-to,black,ultra thick,dashed] (HOLTFF2) -- (HOL2);
  \draw [-to,black,ultra thick,dashed] (HOL2) -- (theorem);
  \draw [-to,black,ultra thick] (Beagle.south) to [out=270,in=270] (theorem.south);
\end{tikzpicture}


\begin{itemize}
\item fonctionnement de la traduction
\item lien vers des papiers expliquant les étapes qui sont déjà dans la
  littérature
\end{itemize}


\subsection{Preuve de correction}

\begin{thm}(Correction de la traduction)
\\Si le problème traduit est un théorème alors le problème dans sa forme original est un théorème.
\end{thm}

\begin{proof}(Idée de la preuve)
\\La traduction est correcte car chaque étape de la traduction est correcte [voir présentation de la traduction]. Cette affirmation est soutenue par l'expérience [voir expérience avec Beagle].
\end{proof}


\subsection{Nouveaux aspects}

\subsubsection{Monomorphisation}


