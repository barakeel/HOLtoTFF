\section{Une traduction prouvée de la logique d'ordre supérieur vers la
  logique du premier ordre}
\label{sec:traduction}

Dans cette partie, nous expliquons notre traduction prouvée des buts
\holfour vers le format \tff, c'est-à-dire la logique du premier ordre
avec arithmétique. Cela se fait par étapes de différentes sortes:
\begin{itemize}
\item des étapes effectivement de traduction de l'ordre supérieur vers
  le premier ordre (notamment la monomorphisation, le $\lambda$-lifting
  et la suppression des fonctions passées en arguments d'autres
  fonctions);
\item d'autres étapes intermédiaires pour tenir compte du fait que l'on
  veut donner le but à un prouveur du premier ordre: il résout un
  problème de satisfiabilité (et non de prouvabilité), attendu en forme
  normale conjonctive présentée sous forme d'une liste de clauses, et
  faisant intervenir certains types prédéfinis.
\end{itemize}
Nous présentons dans un premier temps les différentes étapes de la
traduction, avant de détailler notre algorithme utilisé pour l'une
d'entre elles: la monomorphisation.

\subsection{Les étapes de la traduction}
La traduction est une succession de petites étapes, chacune étant
prouvée correcte. Nous présentons ici l'ordre de ces étapes, en
précisant quel outil a été utilisé pour les implanter. Les étapes
concernant réellement le passage au premier ordre sont notées en gras.
\begin{enumerate}
\item \textbf{Monomorphisation}: Il est souvent admis que pour des raisons d'efficacité, il est préférable d'effectuer cet étape en premier. La
  partie~\ref{sec:traduction:nouveautes} décrit notre algorithme de
  monomorphisation.
  \item Négation de la conclusion: la prouvabilité du but est
    équivalente à la non-satisfiabilité de la formule donnée au
    prouveur.
  \item Mise sous forme normale conjonctive: cette étape est réalisée
    par la fonction \textsf{CNF$\_$CONV} implantée en \holfour par Joe
    Hurd en octobre 2001.
  \item \textbf{$\lambda$-lifting}: cette étape consiste à éliminer les
    variables libres apparaissant dans des définitions locale. Elle est
    implantée de la même manière que dans \cite{Bohme12}. Cette étape
    auraient pu être remplacés par l'utilisation de combinateurs
    \cite{Hurd03}. Ces deux approches ont été testés dans le
    développement de \sledgehammer \cite{MengP08}.
  \item Élimination des arguments booléens qui ne sont pas des
    variables: le format \tff propose un type $\$o$ pour désigner le
    type de retour des prédicats. Il ne peut être utilisé pour les
    arguments booléens des fonctions; leur type est donc remplacé par un
    nouveau type \verb!bool! \cite{MengP08}.
  \item Mise sous forme d'un ensemble de clauses par élimination  des quantificateurs existentiels et distribution des quantificateurs universels par rapport au conjonctions.
  \item \textbf{Défonctionalisation}: cette étape consiste à éliminer
    les fonctions passées en arguments à d'autres
    fonctions~\cite{Hurd03,MengP08}.
    Les fonctions arithmétiques sont néanmoins conservées, puisqu'elles
    font partie de notre langage cible. Elle est réalisée en fin de
    traduction afin de limiter son application: par exemple,
    l'élimination de l'ordre supérieur dans la formule $\exists f.\
    \forall x.\ f\ x = 0$ aura déjà été réalisée par les étapes
    précédentes.
  \item Case-split des variables booléennes liées.
  \item Correspondance entre types entiers: le type \holfour \verb!num!
    représentant les entiers positifs est déplié en un entier \verb!int!
    et la propriété qu'il est positif.
\end{enumerate}

Chaque étape étant prouvée correcte en \holfour, il en résulte qu'une
preuve d'un problème traduit constitue également une preuve du problème
originel.


\subsection{Algorithme de monomorphisation}
\label{sec:traduction:nouveautes}

Cette étape consiste à instancier les variables de type apparaissant dans
les théorèmes prouvés en amont par plusieurs types concrets, le format
\tff n'ayant pas de polymorphisme. La difficulté est de trouver les
types concrets avec lesquels instancier ces variables tout en conservant
la prouvabilité; cette étape est
incomplète~\cite{DBLP:conf/frocos/BobotP11}, mais des heuristiques
donnent de bonnes performances~\cite{DBLP:conf/cade/BlanchetteP13}.

La correction de la monomorphisation est simple à établir, puisqu'il
s'agit d'une ou plusieurs instanciations de variables de type
(implicitement) universellement quantifiées.

Cette partie présente l'heuristique développée pour la traduction de
\holfour vers \tff. Nous présentons d'abord le problème sur un exemple
(qui servira dans toute la suite de cette partie), avant d'expliquer
notre algorithme de monomorphisation. Ses performances seront évaluées
dans la partie~\ref{sec:experiences}.


\subsubsection{Exemple}
\label{sec:traduction:nouveautes:exmp}

Supposons prouvés les théorèmes $\forall x:a.\ C\ x \Rightarrow D\ x$ et
$\forall x:b.\ C\ x$ où $a$ et $b$ sont des variables de type
(implicitement quantifée universellement). On cherche à démontrer le but
$D\ 42$. Les instanciations cherchées sont la substitution $\{a \mapsto
\text{\verb!int!}\}$ pour le premier théorème et la substitution $\{b
\mapsto \text{\verb!int!}\}$ pour le second théorème.

Nous illustrerons sur cet exemple les étapes de monomorphisation
décrites dans la suite de cette partie.

\subsubsection{Graphe de dépendance}
\label{sec:traduction:nouveautes:graphe}
Cette partie présente de manière théorique notre algorithme de
monomorphisation.\\

Étant donné un problème, nous définissons un graphe de dépendance comme suit.
\begin{mydef} (Graphe de dépendance)
Un graphe de dépendance est constitué de nœuds et de deux types
d'arêtes.
\begin{enumerate}
\item[$\bullet$] Les nœuds sont les constantes présentes dans la liste
  de théorèmes et dans le but, associées à leur type. Deux mêmes
  constantes sont représentées par des nœuds différents si elles
  proviennent de deux propositions (théorèmes ou buts) différentes.
\item[$\bullet$] Les arêtes de {\color{blue}substitution} sont orientées
  et relient deux
  constantes vérifiant les propriétés suivantes:
  \begin{itemize}
  \item la constante origine n'appartient pas au but (on rappelle que
    l'on ne cherche à instancier que les variables de type des théorèmes
    prouvés en amont);
  \item elles ont le même nom;
  \item le type du destinataire est une instance du type de l'origine.
  \end{itemize}
\item[$\bullet$] Les arêtes de {\color{green}partage} sont non
  orientées et relient deux constantes
  vérifiant les propriétés suivantes:
  \begin{itemize}
  \item elles appartiennent à un même théorème;
  \item elles partagent une même variable de type.
  \end{itemize}
\end{enumerate}
\end{mydef}

Par convention, dans tous les exemples de graphes de dépendance donnés
ci-dessous, nous représentons les constantes d'une même proposition sur
une même ligne.

\begin{figure}
\begin{tabularx} {\textwidth}{ X  X }
\begin{center}
\begin{tikzpicture}[node distance = 3cm]
  \node [cloud, fill=white,node distance = 3cm] (c11) 
  {C: $a \rightarrow bool$};
  \node [cloud, fill=white, right of=c11,node distance = 4cm] (c12) 
  {D : $a \rightarrow bool$};
  \node [cloud, fill=white, below of=c11,node distance = 1.5cm] (c21) {C: $b \rightarrow bool$};
  \node [cloud, fill=white, below of=c12,node distance = 3cm] (c32) 
  {D: $\text{\verb!int!} \rightarrow \text{\verb!bool!}$};
  \draw[-to,blue,ultra thick](c11) -- (c21);
  \draw [-to,blue,ultra thick] (c21) -- (c11);
  \draw [-to,blue,ultra thick] (c12) -- (c32);
  \draw [green,ultra thick] (c11) -- (c12);
\end{tikzpicture}
\end{center}
&
\begin{center}
\begin{tikzpicture}[node distance = 3cm)]
  \node [cloud, fill=white,node distance = 3cm] (c11) 
  {C: $a$};
  \node [cloud, fill=white, right of=c11,node distance = 3cm] (c12) 
  {D : $a$};
  \node [cloud, fill=white, below of=c11,node distance = 1.5cm] (c21) {C: $b$};
  \node [cloud, fill=white, below of=c12,node distance = 3cm] (c32) 
  {D: $\text{\verb!int!}$};
  \draw[-to,blue,ultra thick](c11) -- (c21);
  \draw [-to,blue,ultra thick] (c21) -- (c11);
  \draw [-to,blue,ultra thick] (c12) -- (c32);
  \draw [green,ultra thick] (c11) -- (c12);
\end{tikzpicture}
\end{center}
\\
\end{tabularx}
\caption {Graphe de dépendance de l'exemple de la partie~\ref{sec:traduction:nouveautes:exmp} (à gauche) et un graphe équivalent (à droite)}
\label{fig:dependances}
\end{figure}

L'idée de l'algorithme est d'appliquer une succession de transformations
au graphe de dépendence, jusqu'à l'obtention d'un point fixe (s'il
existe) nous donnant l'instanciation voulue. Nous définissons tout
d'abord deux types de transformations sur un graphe de dépendence, une
pour chaque type d'arête.

\begin{mydef} (Transformation de substitution) La transformation de
  substitution d'un graphe de dépendence $G$ pour l'arête de
  substitution $\sigma : c \to c'$ instancie les types de $c$ à l'aide  des substitutions induites par $\sigma$. Ces types sont ensuite ajoutés au nœud $c$.
\end{mydef}

\begin{figure}
\begin{center}
\begin{tikzpicture}[node distance = 3cm, auto]
  \node [cloud, fill=white,node distance = 3cm] (c11) 
  {C: $a$};
  \node [cloud, fill=white, right of=c11,node distance = 3cm] (c12) 
  {D: $a$ , $\color{blue} \text{\verb!int!}$};
  \node [cloud, fill=white, below of=c11,node distance = 1.5cm] (c21) {C: b};
  \node [cloud, fill=white, below of=c12,node distance = 2.5cm] (c32) 
  {D: $\text{\verb!int!}$};
  \draw[-to,blue,ultra thick](c11) -- (c21);
  \draw [-to,blue,ultra thick] (c21) -- (c11);
  \draw [-to,blue,ultra thick] (c12) -- (c32);
  \draw [green,ultra thick] (c11) -- (c12);
\end{tikzpicture}
\end{center}
\caption{Transformation de substitution associée à l'arête de substitution de droite}
\label{fig: substitutions}
\end{figure}

\begin{mydef} (Transformation de partage) La transformation de
  partage d'un graphe de dépendence $G$ pour l'arête de
  partage $\rho : c \leftrightarrow c'$ et l'arête de substitution
  $\sigma : c' \to c''$ instancie les types de $c$ à l'aide des substitutions induites par $\sigma$. Ces types sont ensuite ajoutés au nœud $c$.
\end{mydef}

\begin{figure}
\begin{center}
\begin{tikzpicture}[node distance = 3cm, auto]
  \node [cloud, fill=white,node distance = 3cm] (c11) 
  {C: $a$ , $\color{green} \text{\verb!int!}$};
  \node [cloud, fill=white, right of=c11,node distance = 3cm] (c12) 
  {D: $a$};
  \node [cloud, fill=white, below of=c11,node distance = 1.5cm] (c21) {C: b};
  \node [cloud, fill=white, below of=c12,node distance = 2.5cm] (c32) 
  {D: $\text{\verb!int!}$};
  \draw[-to,blue,ultra thick](c11) -- (c21);
  \draw [-to,blue,ultra thick] (c21) -- (c11);
  \draw [-to,blue,ultra thick] (c12) -- (c32);
  \draw [green,ultra thick] (c11) -- (c12);
\end{tikzpicture}
\end{center}
\caption{Transformation de partage associée à l'arête de partage (au centre) et à l'arête de substitution de droite}
\label{fig: substitutions}
\end{figure}

Ces deux transformations sont appliquées simultanément afin de
transformer un graphe de dépendance. Cette transformation est illustrée
par l'exemple de la \textsc{Figure}~\ref{fig:etape_de_monomorphisation}.

\begin{mydef} (Transformation d'un graphe)
La transformation $\mathcal{T}$ d'un graphe de dépendance $G$ est définie comme l'application de toutes les transformations de substitution et de partage possibles à ce graphe de façon simultanée.
\end{mydef}

\begin{figure}
\begin{center}
\begin{tikzpicture}[node distance = 3cm, auto]
  \node [cloud, fill=white,node distance = 3cm] (c11)
  {C: $a$,{\color{blue} $b$},{\color{green} $\text{\verb!int!}$}};
  \node [cloud, fill=white, right of=c11,node distance = 4cm] (c12) 
  {D: $a$,{\color{green} $b$},{\color{blue} $\text{\verb!int!}$}};
  \node [cloud, fill=white, below of=c11,node distance = 1.5cm] (c21) {C: $b$,{\color{blue} $a$}};
  \node [cloud, fill=white, below of=c12,node distance = 2.5cm] (c32) 
  {D: $\text{\verb!int!}$};
  \draw[-to,blue,ultra thick](c11) -- (c21);
  \draw [-to,blue,ultra thick] (c21) -- (c11);
  \draw [-to,blue,ultra thick] (c12) -- (c32);
  \draw [green,ultra thick] (c11) -- (c12);
\end{tikzpicture}
\end{center}
\caption{Une itération de la transformation $\mathcal{T}$ sur le graphe de notre exemple}
\label{fig:etape_de_monomorphisation}
\end{figure}


\begin{mydef} (Redondance) Dans un graphe, une arête de substitution est
  redondante si elle a la m\^eme origine et induit la m\^eme
  substitution qu'une arête de substitution interne au nœud de son
  origine. De telles arêtes peuvent être retirées du graphe pour obtenir
  un graphe équivalent.
\end{mydef}

La notion de redondance est illustrée par la \textsc{Figure}~\ref{fig:redondance}, dans laquelle nous avons ajouté des arêtes internes (en noir).

\begin{figure}
\begin{center}
\begin{tikzpicture}[node distance = 3cm, auto]
  \node [cloud, fill=white,node distance = 3cm] (c11)
  {C: $a$, $b$, $\text{\verb!int!}$};
  \node [cloud, fill=white, right of=c11,node distance = 4cm] (c12)
  {D: $a$, $b$, $\text{\verb!int!}$};
  \node [cloud, fill=white, below of=c11,node distance = 1.5cm] (c21) {C: $b$,$a$};
  \node [cloud, fill=white, below of=c12,node distance = 2.5cm] (c32) 
  {D: $\text{\verb!int!}$};
  \draw [-to,gray,ultra thick] (c11)  to [out=260,in=100] (c21);
  \draw [-to,blue,ultra thick] (c21) to [out=80,in=280] (c11);
  \draw [-to,black,ultra thick] (c11) to [out=130,in=75] (c11);
  \draw [-to,black,ultra thick] (c12) to [out=130,in=25] (c12);
  \draw [-to,gray,ultra thick, ] (c12) -- (c32);
  \draw [green,ultra thick] (c11) -- (c12);
\end{tikzpicture}
\end{center}
\caption{Elimination des arêtes redondantes (en gris)}
\label{fig:redondance}
\end{figure}


Les transformations successives d'un graphe de dépendance ne terminent
pas toujours (ce qui rejoint le fait que la monomorphisation est
incomplète). Nous donnons maintenant un exemple de graphe de dépendance
pour lequel les transformations successives font grossir indéfiniment
les types possibles à chaque nœud.

\begin{thm} (Circuit)
Il existe un graphe de dépendance pour lequel la transformation
$\mathcal{T}$ n'a pas de point fixe.
\end{thm}
\begin{proof}(Contre-exemple)

  La transformation $\mathcal{T}$ n'a pas de point fixe pour le graphe suivant:

\begin{tabularx}{\textwidth}{ X X }

\noindent \textit{Graphe initial.}
\begin{center}
\begin{tikzpicture}[node distance = 3cm, auto]
  % Place nodes
  \node [cloud, fill=white,node distance = 3cm] (c11)
  {A: $a$};
  \node [cloud, fill=white, below of=c11,node distance=1cm] (c12)
  {A: $a \rightarrow n$};
  \draw[-to,blue,ultra thick](c11) to [in=0,out=0] (c12);
  \draw [green,ultra thick] (c11) -- (c12);
\end{tikzpicture}
\end{center}
&
\noindent \textit{Etape 1.}
\begin{center}
\begin{tikzpicture}[node distance = 3cm, auto]
  % Place nodes
  \node [cloud, fill=white,node distance = 2cm] (c11)
  {A: $a, {\color{red} a \rightarrow n}$};
  \node [cloud, fill=white, below of=c11,node distance = 1.2cm] (c12) {A: $a \rightarrow n$, ${\color{red} (a \rightarrow n) \rightarrow n}$ };
  \draw[-to,blue,ultra thick](c11) to [in=0,out=0] (c12);
  \draw [green,ultra thick] (c11) -- (c12);
\end{tikzpicture}
\end{center}
\end{tabularx}

\noindent \textit{Etape 2.}
\begin{center}
\begin{tikzpicture}[node distance = 3cm, auto]
  % Place nodes
  \node [cloud, fill=white,node distance = 1.2cm] (c11)
  {A: $a$,
  {\color{red} $a \rightarrow n$},
  {\color{orange} $(a \rightarrow n) \rightarrow n$}
  };
  \node [cloud, fill=white, below of=c11,node distance = 1.2cm] (c12)
  {A: $a \rightarrow n$,
  {\color{red} $(a \rightarrow n) \rightarrow n$},
  {\color{orange} $((a \rightarrow n) \rightarrow n) \rightarrow n$}
  };
  \draw[-to,blue,ultra thick](c11) to [in=0,out=0] (c12);
  \draw [green,ultra thick] (c11) -- (c12);
\end{tikzpicture}
\end{center}
La substitution \{$a \mapsto (a \rightarrow n)$\} peut être appliquée autant de fois que l'on veut à $a$ sans trouver de point fixe.
\end{proof}


$\mathcal{T}$ pouvant ne pas terminer, nous limitons le nombre
d'itérations pour calculer la substitution finale. Cette borne est
déterminée par la conjecture suivante.

\begin{conjecture}
Soit $p$ le nombre maximal d'arêtes de substitution d'un chemin sans circuit de $G$. \\Soit $G_p = \mathcal{T}^p(G)$. On conjecture que:
\begin{itemize}
\item Si $G_p$ n'a pas de circuit interne à ses nœuds et n'a pas d'arête de substitution non redondante à l'extérieur de ses nœuds, alors $G_p$ est un point fixe pour $\mathcal{T}$.
\item Si $G_p$ n'est pas un point fixe pour $\mathcal{T}$, alors pour tout $i>0$, $G_i$ n'est pas un point fixe pour $\mathcal{T}$.
\end{itemize}
\end{conjecture}

La \textsc{Figure}~\ref{fig:point_fixe} montre le point fixe de
$\mathcal{T}$ sur notre exemple.
\begin{figure}
\begin{center}
\begin{tikzpicture}[node distance = 3cm, auto]
  \node [cloud, fill=white,node distance = 3cm] (c11)
  {C: $a$,{\color{red} $b$},{\color{red} $\text{\verb!int!}$}};
  \node [cloud, fill=white, right of=c11,node distance = 4cm] (c12) 
  {D: $a$,{\color{red} $b$},{\color{red} $\text{\verb!int!}$}};
  \node [cloud, fill=white, below of=c11,node distance = 1.2cm] (c21) {C: $b$,{\color{red} $a$},{\color{orange} $\text{\verb!int!}$}};
  \node [cloud, fill=white, below of=c12,node distance = 2.4cm] (c32) {D: $\text{\verb!int!}$};
\end{tikzpicture}
\end{center}
\caption{Point fixe de la transformation $\mathcal{T}$ après 2 étapes de monomorphisation}
\label{fig:point_fixe}
\end{figure}

\begin{remark}
Le fait que l'on ait obtenu un point fixe après 2 étapes dans la
\textsc{Figure}~\ref{fig:point_fixe} vérifie la conjecture ci-dessus car le nombre maximal d'arêtes de substitution d'un chemin sans circuit du graphe initial (\textsc{Figure}~\ref{fig:dependances}) est de 2.
\end{remark}

\subsubsection{Création de nouveaux théorèmes à partir d'un graphe de dépendance}
\label{sec:traduction:nouveautes:conclusion}

D'un noeud du graphe, nous pouvons extraire des arêtes de substitutions (internes ou externes) dont l'origine est ce noeud lesquels induisent un ensemble de substitutions pour ce noeud.

D'un ensemble de noeuds appartenant à un même théorème, on peut déduire un ensemble de substitutions en combinant les ensembles de subtitutions correspondants à chaque noeud.

Finalement nous instancions chaque théorème à l'aide de leur ensemble de substitutions correspondant et créons ainsi de nouveaux théorèmes susceptibles d'être utilisés par un prouveur ne supportant pas les types polymorphes.






\subsubsection{Implantation}
\label{sec:traduction:nouveautes:implantation}
Nous présentons maintenant quelques détails pratiques pour implanter notre traduction.

\paragraph{Pour gérer les cas où la transformation sur le graphe de
  dépendance n'a pas de point fixe,} nous avons limité à 15 la création
de nouveaux théorèmes, ce qui limite du même coup le nombre d'itérations
que le programme effectue avant de renvoyer le graphe de dépendance
final.

\paragraph{Lors de la monomorphisation,} un graphe de dépendance est
implanté par une liste de listes, chaque liste contenant les constantes
d'un même théorème ainsi que les constantes créées par la répétition de
la transformation. Les arêtes ne sont pas représentées, car toutes les
constantes d'un même théorème sont instanciées avec les substitutions
générées par ces mêmes constantes, ce qui théoriquement revient à
considérer que les constantes d'un même théorème sont liés 2 à 2 par une
arête de partage.

\paragraph{Lors de l'impression,} des dictionnaires injectifs, faisant
correspondre les variables et les constantes de \holfour avec des
variables d'un fichier \tff, sont créés. Ceci augmente la sûreté de
l'impression et permettra de rejouer ensuite la preuve (voir
partie~\ref{sec:traces}). De plus, les types doivent être traduits pour
correspondre à la distinction au premier ordre entre variables,
fonctions et prédicats.


