\section{Travaux similaires et futurs}
\label{sec:similaires}

\paragraph{L'interaction sceptique entre assistants de preuve et
  prouveurs automatiques externes}
a déjà été largement étudiée, pour différents types d'assistants de
preuve -- basés sur la théorie des
types~\cite{DBLP:conf/cpp/ArmandFGKTW11,DBLP:conf/types/Besson06} ou sur
la logique d'ordre supérieur~\cite{Paulson10,KaliszykU12} -- et
différents types de prouveurs automatiques -- du premier
ordre~\cite{Paulson10}, de Satisfiabilité Modulo
Théories~\cite{Paulson10,DBLP:conf/cpp/ArmandFGKTW11,Bohme12}, de
terminaison~\cite{KaliszykU12},
d'arithmétique~\cite{DBLP:conf/types/Besson06}\dots. L'interaction
présentée dans cet article est destinée en particulier à être utilisée
avec des prouveurs du premier ordre \textbf{avec arithmétique}, ce dont
il a fallu tenir compte pour utiliser au mieux cette capacité du
prouveur externe (partie~\ref{sec:traduction:nouveautes}).

\paragraph{Une traduction prouvée de l'ordre supérieur vers le premier
  ordre}
est déjà implantée pour la plupart des applications citées ci-dessus.
Notre traduction reprend les idées présentée dans ces travaux
($\lambda$-lifting, traitement de l'arithmétique non
linéaire~\cite{Bohme12}, instantiation des arguments booléens), mais en
réalisant un traitement particulier pour l'arithmétique linéaire comme
expliqué ci-dessus, ainsi qu'une monomorphisation à l'aide de points
fixes (partie~\ref{sec:traduction:nouveautes}). Si les expériences
présentées dans cet article montrent l'efficacité absolue de notre
algorithme de monorphisation, nous nous réservons comme travaux futurs
une comparaison avec les algorithmes de monomorphisation existants. Un
point particulièrement intéressant serait une comparaison avec une
traduction vers le format \tffone proposé récemment par Blanchette et
Paskevich~\cite{DBLP:conf/cade/BlanchetteP13}.

\paragraph{La vérification de traces} présentée ici constitue un premier
prototype basé sur l'idée de rejouer la preuve à l'aide des prouveurs
automatiques internes à \holfour, comme c'est le cas pour la tactique
\sledgehammer~\cite{Paulson10}. Bien que beaucoup moins efficace pour
l'instant, elle illustre l'aspect selon lequel notre tactique permet de
combiner raisonnement propositionnel et arithmétique: la vérification de
traces peut se faire à l'aide des tactiques \metistac et \coopertac,
mais sans nécessité d'une tactique interne combinant les deux.

