\section{Introduction}

Pour bénéficier de l'état de l'art en matière d'automatisation, de
nombreux assistants de preuve font appel à des prouveurs automatiques,
qu'ils soient internes~\cite{Hurd05,Lescuyer11} ou
externes~\cite{Paulson10,DBLP:conf/cpp/ArmandFGKTW11}, afin de résoudre
certains de leurs buts. Si un prouveur interne confère une autonomie à
l'assistant de preuves dans lequel il est implanté et peut donc offrir
des garanties de complétude -- on peut établir formellement que le
prouveur sera capable de prouver tout ce qui est prouvable --
l'utilisation de prouveurs externes est souvent plus simple -- car il
suffit de vérifier ses réponses \emph{a posteriori} et non de montrer la
correction complète du prouveur -- plus efficace -- car il peut être
très optimisé sans que cela complique la vérification de ses réponses --
et également plus générique -- on peut définir une interface entre
assistants de preuves et prouveurs automatiques indépendante d'un
pouveur particulier. Les deux manières de procéder sont généralement
complémentaires: les prouveurs internes sont utilisés afin de vérifier
les traces générées par les prouveurs externes.

Nous présentons ici une traduction de la logique de l'assistant de
preuve \holfour vers la logique du premier ordre avec arithmétique
(codée dans le format \tff), destinée à pouvoir décharger certains buts
de \holfour vers des prouveurs automatiques du premier ordre. Afin de ne
pas compromettre la cohérence de \holfour:
\begin{itemize}
\item en amont, la traduction est entièrement prouvée correcte en
  \holfour;
\item en aval, un premier vérificateur permet de rejouer les preuves
  éventuellement fournies par le prouveurs automatique (au format \tff)
  en \holfour à l'aide de ses prouveurs internes, notamment
  \metis~\cite{Hurd05} pour la logique du premier ordre et \cooper pour
  l'arithmétique.
\end{itemize}
L'utilisation du format \tff générique rend la traduction et la
vérification de preuves indépendantes du prouveur externe utilisé, et
offre donc la possibilité d'en utiliser plusieurs en parallèle ou en
fonction des besoins.

Cette traduction s'appuie sur les traductions déjà existantes, notamment
celles implantées pour la tactique \sledgehammer~\cite{Paulson10} en
\isabellehol, mais présente également de nouveaux concepts, notamment en
ce qui concerne la \emph{monomorphisation} (c'est-à-dire la suppression
de la quantification sur les variables de type), où nous avons cherché à
donner un algorithme simple mais néanmoins efficace. Elle permet
également de gérer à la fois les quantificateurs du premier ordre et
l'arithmétique.

Cette traduction générique est utilisée pour implanter la nouvelle
tactique \beagletac (présentée en \textsc{Figure~\ref{fig:beagletac}}),
faisant appel au récent prouveur du premier ordre avec arithmétique
\beagle~\cite{DBLP:conf/cade/BaumgartnerW13} (sans vérification de
preuves dans la version actuelle). Bien qu'offrant moins de garantie de
correction que \metistac, la tactique faisant appel au prouveur du
premier ordre interne à \holfour, cette première version de \beagletac
est plus expressive car elle combine raisonnement propositionnel avec
arithmétique. Des expériences montrent également l'efficacité de la
première version de cette tactique. Il s'agit ici d'une des premières
utilisations de \beagle, fournissant ainsi un ensemble de tests
conséquent et montrant son efficacité sur des problèmes concrets.

\begin{figure}[!h]
\begin{center}
\begin{tikzpicture}[node distance = 2cm, auto]
  % Place nodes
  \node [cloud] (conjecture) {conjecture};
  \node [block, right of=conjecture,  node distance=3cm] (HOL1)
  {traduction};
  \node [block, right of=HOL1, node distance=3cm] (TFF1)
  {but};
  \node [block, right of=TFF1, node distance=3cm, yshift=-1cm] (Beagle)
  {preuve automatique};
  \node [block, below of=TFF1, node distance=2cm] (TFF2)
  {trace de preuve};
  \node [block, below of=HOL1, node distance=2cm] (HOL2)
  {preuve rejouée};
  \node [cloud, below of=conjecture, node distance=2cm] (theorem)
  {théorème};
  \node [label=\holfour, draw=black, ultra thick,
  fit=(conjecture) (theorem) (HOL1) (HOL2)] {};
  \node [label=Interface \tff, draw=black, ultra thick, fit=(TFF1) (TFF2)]
  {};
  \node [label=\beagle, draw=black, ultra thick, fit=(Beagle)] {};
  % Draw edges
  \draw [-to,black,ultra thick] (conjecture) -- (HOL1);
  \draw [-to,black,ultra thick] (HOL1) -- (TFF1);
  \draw [-to,,black,ultra thick] (TFF1) -- (Beagle);
  \draw [-to,black,ultra thick,dashed] (Beagle) -- (TFF2);
  \draw [-to,black,ultra thick,dashed] (TFF2) -- (HOL2);
  \draw [-to,black,ultra thick,dashed] (HOL2) -- (theorem);
  \draw [-to,black,ultra thick] (Beagle.south) to       [out=230,in=310] (theorem.south);
\end{tikzpicture}
\caption{La tactique \beagletac: la version courante utilise \beagle
  comme un oracle (flèches pleines), mais ses traces peuvent être
  rejouées à l'aide de \metistac et \coopertac (flèches pointillées).}
\label{fig:beagletac}
\end{center}
\end{figure}

Le code présenté dans cet article est disponible à l'adresse
\url{https://github.com/barakeel/HOLtoTFF}. En particulier, les exemples
présentés dans cet article sont disponibles dans le fichier
\verb!paperexample.sml!.


\subsection{Assistants de preuves et prouveurs automatiques}

Les assistants de preuve sont des logiciels basés sur un petit noyau
implantant un vérificateur pour une logique très expressive, permettant
d'énoncer et de prouver de manière très sûre des théorèmes dans une
grande variété de domaines. Au-dessus de ce petit noyau, une interface
utilisateur contenant notamment un ensemble de tactique permet d'établir
les preuves pas à pas. La grande expressivité de la logique implantée
par ces prouveurs limite l'automatisation de la recherche de preuve
possible, demandant ainsi une assez grande expertise pour être utilisés.

\emph{A contrario}, les prouveurs automatiques implantent une logique
moins expressive, mais pour laquelle une recherche de preuve -- bien
qu'incomplète -- est possible. Ils demandent donc moins de travail à
l'utilisateur, mais offrent très peu de garanties de sûreté.

\begin{table}[!h]
\begin{center}
\begin{tabularx}{\textwidth}{ |c|X|X| }
  \hline
  & Assistants de preuve & Prouveurs automatiques\\
  \hline
  Prouveurs & \holfour, \isabellehol, \hollight, \coq, \ldots
  & \beagle, \spass, \vampire, \zthree, \verit \ldots\\
  \hline
  Logique & Ordre supérieur ou Théorie des types & Premier ordre et combinaison de théories\\
  \hline
  Utilisation & Guidage fastidieux de l'utilisateur & Preuve entièrement automatisée\\
  \hline
  Sûreté & Basé sur un petit noyau vérifiable & Code très grand
  difficile à prouver, retourne parfois des traces vérifiables \emph{a
    posteriori}\\
  \hline
\end{tabularx}
\caption{Différences entre assistants de preuve et prouveurs automatiques}
\label{tab:assistants_automatiques}
\end{center}
\end{table}

La \textsc{Table}~\ref{tab:assistants_automatiques} résument leurs
points forts et faibles, soulignant leur complémentarité: une
interaction entre les deux permettrait d'offre à la fois
l'automatisation lorsqu'elle est possible et la sûreté.

Dans cet article, nous nous intéressons plus particulièrement aux
assistants de preuve basés sur la logique d'ordre supérieur, notamment
\holfour, et aux prouveurs du premier ordre avec théories, comme
\beagle. La logique implantée par les deuxièmes étant un sous-ensemble
de celle implantée par les assistants de preuves basés sur la logique
d'ordre supérieur, elle définit une interface commune leur permettant de
communiquer.


\subsection{Interface: le format \tff}

Cette interface est décrite par le format générique \tff\footnote{Le
  format \tff est décrit à l'adresse
  \url{http://www.cs.miami.edu/~tptp/TPTP/TR/TPTPTR.shtml\#TypeSystem}.}
(signifiant ``Typed First-Order Form''), promu par \tptp\footnote{La
  bibliothèque \tptp est disponible à l'adresse
  \url{http://www.cs.miami.edu/~tptp}.} (signifiant ``Thousands of
Problems for Theorem Provers'') regroupant une grande base de données de
problèmes pour prouveurs du premier ordre. Il permet de décrire tout
problème du premier ordre avec arithmétique (entière, rationnelle ou
réelle). Il propose également un format générique pour les traces de
preuves\footnote{Le standard des preuves écrites au format \tff est présenté à l'adresse \url{http://www.cs.miami.edu/~tptp/TPTP/QuickGuide/Derivations.htm})}


L'utilisation d'un format générique permet de ne pas dépendre de
prouveurs en particuliers, comme l'est la traduction décrite dans la
partie~\ref{sec:traduction}.


\subsection{Plan}

Le papier est organisé comme suit. La partie~\ref{sec:traduction}
présente une traduction de la logique d'ordre supérieur vers la logique
du premier ordre, entièrement implantée et prouvée en \holfour. Elle
présente de nouveaux aspects, notamment concernant la monomorphisation
(partie~\ref{sec:traduction:nouveautes}). Cette traduction est utilisée
pour définir la nouvelle tactique \beagletac
(partie~\ref{sec:experiences:beagletac}), dont l'efficacité est ensuite
testée (partie~\ref{sec:experiences:experiences}). La partie
\ref{sec:traces} décrit comment, dans un deuxième temps, vérifier les
traces au format \tff générées par certains prouveurs (dont \beagle).
Enfin, nous discutions des travaux similaires et futurs
(partie~\ref{sec:similaires}) avant de conclure.




