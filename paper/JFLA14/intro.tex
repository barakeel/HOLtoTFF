\section{Introduction}

Pour bénéficier de l'état de l'art en matière d'automatisation, de
nombreux assistants de preuve font appel à des prouveurs automatiques,
qu'ils soient internes~\cite{Hurd05,Lescuyer11} ou
externes~\cite{Paulson10,DBLP:conf/cpp/ArmandFGKTW11}, afin de résoudre
certains de leurs buts. Si un prouveur interne confère une autonomie à
l'assistant de preuves dans lequel il est implanté et peut donc offrir
des garanties de complétude -- on peut établir formellement que le
prouveur sera capable de prouver tout ce qui est prouvable --
l'utilisation de prouveurs externes est souvent plus simple -- car il
suffit de vérifier ses réponses \emph{a posteriori} et non de montrer la
correction complète du prouveur -- plus efficace -- car il peut être
très optimisé sans que cela complique la vérification de ses réponses --
et également plus générique -- on peut définir une interface entre
assistants de preuves et prouveurs automatiques indépendante d'un
pouveur particulier. Les deux manières de procéder sont généralement
complémentaires: les prouveurs internes sont utilisés afin de vérifier
les traces générées par les prouveurs externes.

Nous présentons ici une traduction de la logique de l'assistant de
preuve \holfour vers la logique du premier ordre avec arithmétique
(codée dans le format \tff), destinée à pouvoir décharger certains buts
de \holfour vers des prouveurs automatiques du premier ordre. Afin de ne
pas compromettre la cohérence de \holfour:
\begin{itemize}
\item en amont, la traduction est entièrement prouvée correcte en
  \holfour;
\item en aval, un premier vérificateur permet de rejouer les preuves
  éventuellement fournies par le prouveurs automatique (au format \tff)
  en \holfour à l'aide de ses prouveurs internes, notamment
  \metis~\cite{Hurd05} pour la logique du premier ordre et \cooper pour
  l'arithmétique.
\end{itemize}
L'utilisation du format \tff générique rend la traduction et la
vérification de preuves indépendantes du prouveur externe utilisé, et
offre donc la possibilité d'en utiliser plusieurs en parallèle ou en
fonction des besoins.

Cette traduction s'appuie sur les traductions déjà existantes, notamment
celles implantées pour la tactique \sledgehammer~\cite{Paulson10} en
\isabellehol, mais présente également de nouveaux concepts, notamment en
ce qui concerne la \emph{monomorphisation}, c'est-à-dire la suppression
de la quantification sur les variables de type.

Cette traduction générique est utilisée pour implanter la nouvelle
tactique \beagletac, faisant appel au prouveur du premier ordre avec
arithmétique \beagle~\cite{DBLP:conf/cade/BaumgartnerW13} (sans
vérification de preuves dans la version actuelle). Bien qu'offrant moins
de garantie de correction que \metistac, la tactique faisant appel au
prouveur du premier ordre interne à \holfour, cette première version de
\beagletac est plus expressive car elle combine raisonnement
propositionnel avec arithmétique. Des expériences montrent également son
efficacité: \todo Rapport entre les temps de calcul \beagletac et ceux
\metistac.


\subsection{Assistants de preuves basés sur la logique d'ordre supérieur
et prouveurs du premier ordre}

\todo Chantal

\begin{itemize}
\item différences
\item description de \holfour et \beagle
\end{itemize}


\subsection{Interface: le format \tff}

\todo Chantal

\begin{itemize}
\item description de TFF
\end{itemize}


\subsection{Plan}

Le papier est organisé comme suit.

\todo Chantal
