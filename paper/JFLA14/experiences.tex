\section{Utilisation de \beagle et expériences}
\label{sec:experiences}

\subsection{Présentation de \beagle}
\beagle est un prouveur du premier ordre avec arithmétique très récent,
prenant en entrée le format \tff. Des résultats expérimentaux (sur une
version obsolète) ainsi que la théorie détaillée de \beagle peuvent être
trouvés dans~\cite{DBLP:conf/cade/BaumgartnerW13}.


\subsection{La tactique \beagletac}
\label{sec:experiences:beagletac}

\todo


\subsection{Expériences}
\label{sec:experiences:experiences}

\subsubsection{Software, hardware et tests}
Les tests sont effectués avec la version~$0.7$ de \beagle (\todo:
version de \holfour), sur un processeur deux cœurs cadencé à $2,1$~GHz
avec $3,7$~Go de mémoire vive (\todo 3,7??).

Lors de la construction de \holfour, certains buts sont résolus par la
tactique \metistac. La plupart de ces buts ne font intervenir que du
raisonnement propositionnel, mais certains nécessitent de le combiner
avec un raisonnement arithmétique, auquel cas les lemmes arithmétiques à
utiliser sont fournis à \metistac. Pour les expériences, nous avons
utilisé \beagletac sur $271$ de ces buts, sans donner aucun lemme
arithmétique.


\subsubsection{Résultats}

\todo Temps de calcul, nombre de buts résolus\dots avec \metistac et
avec \beagletac. Conclusion sur l'efficacité comparée de \beagletac et
\metistac, puis celle de \beagle et \metis.

A travers ces exemples, il est difficile de juger de la qualité de \beagle
face à \metis puisque \metis résout l'ensemble de ces buts 
rapidement.
Cependant quelques exemples nous montre que \beagle est plus efficace 


\subsubsection{Impact de la monomorphisation}

\begin{tabularx}{\textwidth}{|X|X|}
\hline
Avant monomorphisation & Après monomorphisation \\
\hline
\begin{tikzpicture}[scale=1.5]
    \slice{0/100*360}
          {70/100*360}
          {70\%}{insatisfiable}{green}
    \slice{70/100*360}
          {84/100*360}
          {14\%}{satisfiable}{red}      
    \slice{84/100*360}
          {91/100*360}
          {7\%}{inconnu}{red}
    \slice{91/100*360}
          {99/100*360}
          {8\%}{time out}{red}
    \slice{99/100*360}
          {100/100*360}
          {1\%}{parsing error}{red}                            
\end{tikzpicture}
&
\begin{tikzpicture}[scale=1.5]
    \slice{0/100*360}
          {80/100*360}
          {80\%}{insatisfiable}{green}
    \slice{80/100*360}
          {81/100*360}
          {1\%}{satisfiable}{red}  
    \slice{81/100*360}
          {86/100*360}
          {5\%}{inconnu, yshift=6}{red}   
     \slice{86/100*360}
           {98/100*360}
           {12\%}{time out}{red}     
     \slice{98/100*360}
           {100/100*360}
           {2\%}{parsing error}{red}               
\end{tikzpicture}
\\
\hline
\end{tabularx}



\subsubsection{Expressivité}

Grâce à notre traduction, \beagletac permet de résoudre des problèmes
combinant d'ordre supérieur avec arithmétique, lorsque le raisonnement
propositionnel est du premier ordre. Cette tactique est donc plus
expressive que \metistac, puisqu'il n'est pas besoin de fournir les
lemmes arithmétiques utilisés.

