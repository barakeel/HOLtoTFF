\section{Utilisation de \beagle et expériences}
\label{sec:experiences}

\subsection{Présentation de \beagle}
\beagle est un prouveur du premier ordre avec arithmétique très récent,
prenant en entrée le format \tff. Des résultats expérimentaux (sur une
version plus ancienne) ainsi que la théorie détaillée de \beagle peuvent être trouvés dans~\cite{DBLP:conf/cade/BaumgartnerW13}.


\subsection{La tactique \beagletac}
\label{sec:experiences:beagletac}

Grâce à cette traduction, nous avons implanté la tactique \beagletac
exportant le but courant transformé par la traduction présentée dans la
partie précédente au format \tff. \beagle est ensuite appelé sur ce but
transformé; lorsqu'il répond que celui-ci est insatisfiable, il est
ajouté comme axiome à \holfour afin de résoudre le but initial (voir
\textsc{Figure}~\ref{fig:beagletac}).

Comme \beagle ne supporte pas l'arithmétique non linéaire, la constante
multiplicative $*$ est traduite par une constante non interprétée
lorsque le terme est non-linéaire. Cette modification devra être retirée
si l'on souhaite utiliser un prouveur qui supporte l'arithmétique non
linéaire.


\subsection{Expériences}
\label{sec:experiences:experiences}

\subsubsection{Software, hardware et tests}
Les tests ont été effectués avec \beagle (version $0.7$) et \holfour (version mai 2013 du dépot \path{https://github.com/mn200/HOL}), sur un processeur deux cœurs cadencé à $2.1$~GHz avec $3.7$~Go de mémoire vive.
Nous avons imposé un timeout de 15 secondes par but à \beagle.

Lors de la construction de \holfour, certains buts sont résolus par la
tactique \metistac. La plupart de ces buts ne font intervenir que du
raisonnement propositionnel, mais certains nécessitent de le combiner
avec un raisonnement arithmétique, auquel cas les lemmes arithmétiques à
utiliser sont fournis à \metistac. Pour les expériences, nous avons
utilisé \beagletac sur $271$ de ces buts, mais sans donner aucun lemme
arithmétique. Afin de mesurer l'impact de notre algorithme de
monomorphisation, nous avons lancé \beagletac avec ou sans
monomorphisation.

\subsubsection{Résultats}

Les résultats en termes de nombre de buts résolus sont présentés dans la
\textsc{Figure}~\ref{fig:resultats}.

% \begin{figure}[H]
% \centering
% \begin{tikzpicture}[scale=1.6,baseline=(current bounding box.center)]
%     \slice{0/100*360}
%           {80/100*360}
%           {80,5\%}{insatisfiable}{green}
%     \slice{80.5/100*360}
%           {81.5/100*360}
%           {1\%}{satisfiable}{red}
%     \slice{81.5/100*360}
%           {89.5/100*360}
%           {8\%}{inconnu, yshift=6}{red}
%      \slice{89.5/100*360}
%            {98.5/100*360}
%            {9\%}{timeout}{red}
%      \slice{98.5/100*360}
%            {100/100*360}
%            {1,5\%}{parsing error}{red}
% \end{tikzpicture}
% \caption{Nombre de buts résolus par \beagletac}
% \label{fig:resultats}
% \end{figure}

\begin{figure}
\noindent \begin{tabularx}{\textwidth}{|X|X|}
\hline
Sans monomorphisation & Avec monomorphisation \\
\hline
\begin{tikzpicture}[scale=1,baseline=(current bounding box.center)]
    \slice{0/100*360}
          {70/100*360}
          {70\%}{insatisfiable}{green}
    \slice{70/100*360}
          {84/100*360}
          {14\%}{satisfiable}{red}
    \slice{84/100*360}
          {91/100*360}
          {7\%}{inconnu}{red}
    \slice{91/100*360}
          {99/100*360}
          {8\%}{timeout}{red}
    \slice{99/100*360}
          {100/100*360}
          {1\%}{parsing error}{red}
\end{tikzpicture}
&
\begin{tikzpicture}[scale=1,baseline=(current bounding box.center)]
    \slice{0/100*360}
          {80/100*360}
          {80.5\%}{insatisfiable}{green}
    \slice{80.5/100*360}
          {81.5/100*360}
          {1\%}{satisfiable}{red}
    \slice{81.5/100*360}
          {89.5/100*360}
          {8\%}{inconnu, yshift=6}{red}
     \slice{89.5/100*360}
           {98.5/100*360}
           {9\%}{timeout}{red}
     \slice{98.5/100*360}
           {100/100*360}
           {1.5\%}{parsing error}{red}
\end{tikzpicture}
\\
\hline
\end{tabularx}
\caption{Proportion de buts résolus par \beagletac sans et avec monomorphisation}
\label{fig:resultats}
\end{figure}

N'étant pas complet, \beagle peut parfois répondre ``inconnu'' s'il ne
sait pas trouver la solution du problème. En revanche, les
``satisfiable'' (dûs à un manque d'information sur les constantes de
\holfour passées à \beagle) et ``parsing error'' (dûs à un fichier
problème trop grand ou à une erreur dans l'impression) correspondent à
des erreurs dans notre utiliation de \beagle.

La \textsc{Table~\ref{tab:temps_calcul}} présente les résultats en
termes de temps de calcul: on compare le temps mis par \beagletac
(décomposé en le temps de la traduction, celui de l'impression et celui
mis par \beagle) au temps mis par \metistac. La tactique \metistac est
plus rapide que \beagletac. La traduction en elle-même prend plus de
temps que \metistac, mais c'est surtout \beagle qui apparaît comme le
facteur de temps limitant. Comme nous le verrons plus bas, cela est
principalement dû aux faiblesses de notre traduction actuelle. Il faut
rappeler également que \beagle, contrairement à \metis, doit effectuer
lui-même le raisonnement arithmétique, mais cela n'a que peu d'influence
comme montré ci-dessous.

\begin{table}[H]
\begin{tabularx}{\textwidth}{|X|X|X|X|X|}
\hline
  \beagletac & Traduction & Impression & \beagle & \metistac\\ \hline
  4,55 & 0,82 & 0,18 & 3,55 & 0,11 \\ \hline
\end{tabularx}
\caption{Temps moyens de résolution (en secondes)}
\label{tab:temps_calcul}
\end{table}

Nous allons maintenant nous intéresser aux différents aspects mis en
valeur par ces expériences.


\paragraph{Sur des problèmes d'ordre supérieur}

La \textsc{Table}~\ref{tab:temps_calcul_premier_ordre_ordre_sup} indique
la proportion de problèmes résolus et les temps de calcul en séparant
les problèmes qui sont initialement du premier ordre avec ceux d'ordre
supérieur. Si \beagletac a un comportement raisonnable sur les buts qui
sont déjà du premier ordre, il n'arrive en revanche qu'à résoudre la
moitié des buts contenant de l'ordre supérieur. Notre traduction est
donc moins performante que celle de \metistac (là encore, le résultat
est légèrement biaisé par le fait que \beagle doit gérer
l'arithmétique).

\begin{table}[H]
\begin{tabularx}{\textwidth}{|X|c|c|c|c|c|c|}
\hline
$ $ & Proportion & \beagletac & Traduction & Impression & \beagle & \metistac \\ \hline
Premier ordre & 91,9\% & 2,711 & 0,148 & 0,08 & 2,483 & 0,13 \\ \hline
O. supérieur & 55,8\%  & 11,07 & 3,2 & 0,51 & 7,36 & 0,04 \\ \hline
\end{tabularx}
\caption{Temps moyens de résolution de problème du premier ordre et d'ordre supérieur (en secondes)}
\label{tab:temps_calcul_premier_ordre_ordre_sup}
\end{table}
 
\paragraph{Sur des problèmes polymorphes}

La \textsc{Table}~\ref{tab:temps_calcul_mono_poly} indique
la proportion de problèmes résolus et les temps de calcul en séparant
les problèmes qui sont initialement monomorphent avec ceux qui sont
polymorphes. On constate qu'aussi bien la proportion de buts résolus que
les temps de calcul sont voisins dans les deux cas, ce qui indique que
notre algorithme de monomorphisation est efficace et bien adapté à notre
traduction actuelle. Sa simplicité se reflète dans le temps mis par la
traduction puisqu'il ne produit pas de délai supplémentaire.

\begin{table}[H]
\begin{tabularx}{\textwidth}{|X|c|c|c|c|c|c|}
\hline
$ $ & Proportion & \beagletac & Traduction & Impression & \beagle & \metistac \\ \hline
Monomorphe & 81,1\% & 5,83 & 1,32 & 0,23 & 4,28 & 0,12\\ \hline
Polymorphe & 79,9\%  & 3,32 & 0,34 & 0,13 & 2,85 & 0,9\\ \hline
\end{tabularx}
\caption{Temps moyen de résolution de problèmes monomorphes et polymorphes (en secondes)}
\label{tab:temps_calcul_mono_poly}
\end{table}

La \textsc{Figure}~\ref{fig:resultats} compare la proportion de buts
résolus sans et avec monomorphisation. On remarque que l'ajout de
l'étape de monomorphisation permet à \beagletac de résoudre $10\%$ de
buts supplémentaires. On constate cependant que cette étape augmente la
taille des buts, faisant ainsi croître le nombre de timeouts. Concernant
l'algorithme en lui-même, un point fixe a été trouvé dans 102 cas des
139 problèmes ayant besoin d'être monomorphisés, soit $73\%$ des cas.

\vspace{1mm}

\paragraph{Sur des problèmes arithmétiques}

La \textsc{Table}~\ref{tab:temps_calcul_arith} indique la proportion de
problèmes résolus et les temps de calcul en séparant les problèmes
contenant de l'arithmétique de ceux qui n'en contiennent pas. Là encore,
la proportion de buts résolus et les temps de calcul sont voisins dans
les deux cas, ce qui montre l'efficacité de notre tactique à résoudre
des buts arithmétique.

\begin{table}[H]
\begin{tabularx}{\textwidth}{|X|c|c|c|c|c|c|}
\hline
$ $ & Proportion & \beagletac & Traduction & Impression & \beagle & \metistac \\ \hline
Non-arithmétique & 81,6\% & 5,91 & 1,51 & 0,25 & 4,15 & 0,08\\ \hline
Arithmétique & 79,6\%  & 3,62 & 0,34 & 0,13 & 3,15 & 0,13\\ \hline
\end{tabularx}
\caption{Temps moyens de résolution de problèmes non-arithmétiques et arithmétiques (en secondes)}
\label{tab:temps_calcul_arith}
\end{table}



% Comme expliqué ci-dessus, parmi les 271 problèmes initialement testés,
% nous avons retenu 65 problèmes contenant au moins un lemme arithmétique.
% Alors que \metistac a besoin de ce lemme pour résoudre le but,
% \beagletac y parvient dans $88\%$ des cas sans l'aide d'aucun lemme.
% \beagletac permet donc de résoudre des problèmes combinant d'ordre
% supérieur avec arithmétique, lorsque le raisonnement propositionnel est
% du premier ordre. Cette tactique est donc plus expressive que \metistac,
% et fournit la première tactique \holfour capable de combiner
% raisonnement au premier ordre et arithmétique.

\paragraph {Conclusion}
\beagletac est à l'heure actuelle bien moins rapide que
\metistac. Outre les temps nécessaires à l'ouverture et la fermeture de
fichiers, \beagletac pèche sur les problèmes d'ordre supérieur, ce qui
indique que notre traduction peut être encore grandement améliorée.
Comme indiqué dans la partie~\ref{sec:traduction}, nous avons
principalement mis l'accent sur la monomorphisation, ce qui se reflète
dans les résultats.

En revanche, \beagletac est plus expressive que \metistac, et fournit la
première tactique \holfour capable de combiner raisonnement au premier
ordre et arithmétique.

Voici quelques directions possibles afin d'améliorer notre traduction:
\begin{itemize}
\item faire une élimination plus fine de l'ordre supérieur~\cite{Bohme12};
\item inférer les lemmes utiles pour le problème à partir d'une
  heuristique sur les constantes~\cite{Paulson10};
\item traduire les types rationnels et réels de \holfour vers les types
  \tff correspondants (pour l'instant, seul l'arithmétique entière est
  gérée).
\end{itemize}

