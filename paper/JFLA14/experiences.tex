\section{Expériences avec \beagle}
\subsection{Software, hardware et tests}
\beagle est en cours de dévellopement et nos tests ont utilisés la version 0.6 de \beagle. Le matériel utilisé lors de ces expériences est un ordinateur portable de processeur AMD Athlon(tm) II P360 Dual-Core avec 3,7 GB de mémoire vive. Les problèmes testés sont les 271 premiers problèmes que \metis résout lors de la construction de \holfour.

\subsection{Présentation de \beagle}


\subsection{Impact de la monomorphisation}

\begin{tabularx}{\textwidth}{|X|X|}
\hline
Avant monomorphisation & Après monomorphisation \\
\hline
\begin{tikzpicture}[scale=1.5]
\diagrammone                
\end{tikzpicture}
&
\begin{tikzpicture}[scale=1.5]
\diagrammtwo 
\end{tikzpicture}
\\
\hline
\end{tabularx}

\subsection{Efficacité de \beagle}
A travers ses exemples, il est difficile de juger de la qualité de \beagle
face à \metis puisque \metis résout l'ensemble de ses buts 
rapidement.
Cependant quelques exemples nous montre que \beagle est plus efficace 
De plus \beagle combine raisonnement du premier ordre et l'arithmétique, ce qui grâce à notre traduction permet à \beagletac de résoudre des problèmes combinant ordre supérieur et arithmétique.

